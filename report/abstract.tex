% \documentclass[crop=false]{standalone}
% \usepackage{standard}
% \begin{document}
  \hrule
  \medskip
  \begin{abstract}
    \itshape
    The restricted Boltzmann machine (RBM) is a network of stochastic units separated into two subsets, the visible and hidden units.
    It only allows undirected interactions between those two subsets.
    These basic requirements result in powerful properties of its mathematical structure.
    This leads to an incredible variety of use cases for RBMs in machine learning, mathematics and physics.
    Especially, for the topic of collaborative filtering it could be shown that RBMs are a state-of-the-art tool.
    In this report I give a short introduction to binary RBMs and their application to collaborative filtering problems using the example of predicting movie ratings made by users.
    Additionally, I write about some hints on how to implement such a system efficiently in C++.
    At the end there are suggestions for further development and investigation.
  \end{abstract}
  \medskip
  \hrule
% \end{document}