\documentclass[crop=false,10pt]{standalone}
\usepackage{standard}

\begin{document}
  \section{Conclusion} % (fold)
  \label{sec:Conclusion}
    RBMs have a simple structure and can be trained with CD which is an efficient algorithm.
    Based on SOURCE they seem to be one of the best known methods for collaborative filtering.
    As said in the introduction this is not the only application.
    RBMs should be used as basic building blocks.
    They are a powerful tool.
    Use them if there is a simple connection to hidden features in your data and if you want to predict something.
    It may be a good idea to insert these into your DNNs as dimensionality reduction.

    Of course one should consider to tweak the explained ideas and algorithms.
    One can use momentum, weight decay and different types of units.
    There are some variants of the contrastive divergence as well.
    According the PAPER even mathematics has not completed the topic of RBMs.
    They seem to be promising in explaining the connection of quantum theory and deep neural networks.
  % section Conclusion (end)
\end{document}