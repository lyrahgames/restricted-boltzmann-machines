\documentclass[crop=false,10pt]{standalone}
\usepackage{standard}

\begin{document}
  \section{The Problem} % (fold)
  \label{sec:the_problem}
    \begin{table*}
      \center
      \caption{%
        The table shows examples of binary ratings for movies made by some imaginary users.
        Every row represents a user and every column a movie.
        The number $0$ is used to point out that the user does not like the movie and $1$ for the opposite.
        The symbol $\times$ is used if there was no rating for the movie by this user.
      }
      \label{tab:problem-example}
      \small
      \documentclass{standalone}

\usepackage{bookman}

\begin{document}
  \sffamily
  \renewcommand{\arraystretch}{1.5}
  \begin{tabular}{lccccc}
    \hline
     & Star Trek & The Matrix & Van Helsing & Harry Potter & The Hobbit \\
    \hline
    \hline
    James T. Kirk & $1$ & $1$ & $\times$ & $0$ & $\times$\\
    Trinity & $\times$ & $1$ & $0$ & $1$ & $1$\\
    Anna Valerious & $\times$ & $\times$ & $1$ & $\times$ & $0$\\
    Severus Snape & $0$ & $1$ & $0$ & $1$ & $0$\\
    Thorin Oakenshield & $1$ & $1$ & $1$ & $\times$ & $0$\\
    \hline
  \end{tabular}
\end{document}
    \end{table*}
    Collaborative filtering as seen in a modern narrow sense basically can be described as a technique to make automatic predictions about interests of users by collecting the preferences or tastes of many users.
    One often refers  to predicting as \enquote{filtering} and to collecting as \enquote{collaborating}.
    The underlying assumption of the collaborative filtering approach is that if two persons have the same opinion on one issue then it is likely that they will also have the same opinion on another issue.
    \cite{WikipediaCollaborativeFiltering}

    For convenience table \ref{tab:problem-example} shows examples for movie ratings made by some imaginary users.
    The entries with $0$ and $1$ determine if a user likes a movie or not and are already known to us.
    They shall be used to predict the unknown values marked with $\times$.
    In the example the users \enquote{James T. Kirk} and \enquote{Thorin Oakenshield} both like the movies \enquote{Star Trek} and \enquote{The Matrix}.
    So one could assume that both users have similar preferences and therefore the user \enquote{James T. Kirk} would also like the movie \enquote{Van Helsing} because the same is already true for the user \enquote{Thorin Oakenshield}.
    Vice versa one may say that the user \enquote{Thorin Oakenshield} does not like the movie \enquote{Harry Potter} since \enquote{James T. Kirk} provided a negative rating for that movie.

    Here the collaborative filtering problem of predicting user ratings for movies shall be solved by using RBMs as it was done in \cite{Hinton2007}.
    For that we have to achieve three main goals that together are forming a basic outline of understanding RBMs and their application to collaborative filtering.
    \begin{description}
      \item[Model:]{%
        Approximately represent probability distributions over different user-movie-ratings.
      }
      \item[Learn:]{%
        Learn an optimal probability distribution in this representation based on some given samples.
      }
      \item[Infer:]{%
        Make predictions for unrated movies.
      }
    \end{description}
\end{document}