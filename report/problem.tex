\documentclass[crop=false,10pt]{standalone}
\usepackage{standard}

\begin{document}
  \section{The Problem} % (fold)
  \label{sec:the_problem}
    Collaborative filtering as seen in a modern narrow sense basically can be described as a method to make automatic predictions about interests of users by collecting the preferences or tastes of many users.
    The underlying assumption of the collaborative filtering approach is that if a person A has the same opinion as a person B on an issue, A is more likely to have B's opinion on a different issue than that of a randomly chosen person.

    To understand the statement of the problem we will use the prediction of movie ratings as done for the \enquote{Netflix Prize} competition.
    In table \ref{tab:problem-example} one can see some examples for these ratings.
    The entries with $0$ and $1$ are already known and shall be used to predict the unknown values with $\times$.
    In the example the users \enquote{James T. Kirk} and \enquote{Thorin Oakenshield} both like the movies \enquote{Star Trek} and \enquote{The Matrix}.
    So one could assume that the user \enquote{James T. Kirk} likes the movie \enquote{Van Helsing} and that the user \enquote{Thorin Oakenshield} does not like the movie \enquote{Harry Potter}.
    \begin{table*}[h]
      \center
      \caption{%
        The table shows examples for ratings of movies made by some users.
        Every row represents a user and every column a movie.
        $0$ stands for \enquote{does not like} and $1$ for \enquote{likes}.
        $\times$ is used if there was no rating for the movie by this user.
      }
      \label{tab:problem-example}
      % \footnotesize
      \documentclass{standalone}

\usepackage{bookman}

\begin{document}
  \sffamily
  \renewcommand{\arraystretch}{1.5}
  \begin{tabular}{lccccc}
    \hline
     & Star Trek & The Matrix & Van Helsing & Harry Potter & The Hobbit \\
    \hline
    \hline
    James T. Kirk & $1$ & $1$ & $\times$ & $0$ & $\times$\\
    Trinity & $\times$ & $1$ & $0$ & $1$ & $1$\\
    Anna Valerious & $\times$ & $\times$ & $1$ & $\times$ & $0$\\
    Severus Snape & $0$ & $1$ & $0$ & $1$ & $0$\\
    Thorin Oakenshield & $1$ & $1$ & $1$ & $\times$ & $0$\\
    \hline
  \end{tabular}
\end{document}
    \end{table*}

    The abstract goals to achieve this can be described as follows.
    \begin{itemize}
      \item Approximately represent probability distributions over ratings
      \item Learn a probability distribution based on some given ratings.
      \item Make predictions for unrated movies for learned parameters.
    \end{itemize}
  % section the_problem (end)
\end{document}